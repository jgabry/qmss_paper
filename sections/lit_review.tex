
\chapter{Literature review}
\label{lit_review}

This chapter draws on the relevant statistical literature to review the theory behind the model applied in the empirical example in Chapter~\ref{cox_katz}. Section \ref{hierarchical} is a brief review of hierarchical models. Gaussian Markov random fields (GMRF) are defined and discussed in \ref{gmrf}. Finally, \ref{star} discusses Bayesian structured additive regression (STAR) models. 

\section{Hierarchical Bayesian models}
\label{hierarchical}

\begin{quote}
{\it Clearly, there is a tension here, between parsimony and realism \dots }

\hfill \shortcite[p.~302]{jackman_bayesian_2009}    
\end{quote}

One of the many challenges of fitting models to data comprising multiple groups is confronting the tradeoff between bias and variance. An analysis that disregards between-group heterogeneity can yield parameter estimates with low variance but high bias. Group-by-group analyses, on the other hand, can reduce bias at the expense of high-variance estimates \shortcite{jackman_bayesian_2009}. While complete pooling or no pooling of data across groups is sometimes called for, nonhierarchical models for hierarchical data tend to underfit or overfit the data \shortcite{gelman_bayesian_2013}. Hierarchical modeling provides a compromise by allowing parameters to vary by group at lower levels of the hierarchical while estimating population-level parameters at higher levels. 

For example, consider a binomial model for the number of survey respondents $y$ in favor of a particular policy. It might be reasonable to estimate separate underlying levels of support $\theta$ for the policy for each of $R$ different geographical regions. However, it is also likely that there is some dependence between the $\theta$s that should be incorporated into the joint probability model. This can be expressed with a prior distribution for the $\theta$s conditional on shared hyperparameters $\phi$, potentially also estimating $\phi$ from the data 

\begin{align*}
y_r | \theta_r &\sim {\rm Bin}(N_r, \theta_r), \quad r = 1, \dots, R \\
\theta_r  | \phi &\sim p(\theta_r | \phi) \\
\phi & \sim p(\phi | \xi)  \\
\vdots
\end{align*}

In theory there is no limit to the number of levels in the hierarchy -- if the value of $\xi$ is not fixed in the example above then it too must be modeled -- however, in practice the data is informative only up to a point (and there are computational challenges to be considered as well). 

Perhaps the most important feature of hierarchical models is that inference for each group-level parameter is informed not only by the group-specific information contained in the data but also by the other groups as well \shortcite{jackman_bayesian_2009}.\footnote{The assumption of exchangeability is important here, but a sufficient discussion of exchangeability is beyond the scope of this thesis. See \citeA{gelman_bayesian_2013}.} This is commonly referred to as borrowing strength. In the example above, for instance, inferences about each $\theta_r$ are informed by $y_r$ but also by $y_{-r}$ through $\phi$.\footnote{The notation $y_{-r}$ is used here to refer to all $y$'s other than $y_r$ (e.g. $y_{-1} =  \{y_2, y_3, \dots, y_R\}$ ). Similar notation is used in later sections as well. } 

The term {\it hierarchical} is a suitable designation for a wide (theoretically unlimited) family of models, from the simple binomial example above to models with regressions at multiple levels. See \citeA{gelman_bayesian_2013} and \citeA{jackman_bayesian_2009} for more thorough and formal introductions to the topic.  




\section{Gaussian Markov random fields}
\label{gmrf}

The prior distributions of particular interest in this thesis are known as Gaussian Markov random field (GMRF) priors. The literature on GMRFs is vast, as they are frequently used in image processing and spatial statistics, however  GMRFs appear only rarely in quantitative social science and \citeA{wawro_designing_2014} is the first example extending the applications of GMRFs to historical social scientific inquiry. Before defining GMRFs it is first necessary to introduce some basic concepts from graph theory, particular the idea of an undirected graph. 

\subsection{Undirected graphs}

 \begin{figure}[htb]
\centering

\textsc{graph A} \hspace{6cm} \textsc{graph B} 

\vspace{.5cm}

\begin{tikzpicture}
\node[obs, fill=DarkSalmon] (r1) {$v_1$};
\node[obs, below left=of r1, fill=DarkSalmon] (r2) {$v_2$};
\node[obs, below=of r2, fill=DarkSalmon] (r3) {$v_3$};
\node[obs, below right=of r3, fill=DarkSalmon] (r4) {$v_4$};
\node[obs, above right=of r4, fill=DarkSalmon] (r5) {$v_5$};
\node[obs, above=of r5, fill=DarkSalmon] (r6) {$v_6$};
\edge [-, color=MidnightBlue, bend right=30] {r1} {r2} ;
\edge [-, color=MidnightBlue, bend left=30] {r1} {r6} ;
\edge [-, color=MidnightBlue] {r2} {r3} ;
\edge [-, color=MidnightBlue, bend right=30] {r3} {r4} ;
\edge [-, color=MidnightBlue, bend right=30] {r4} {r5} ;
\edge [-, color=MidnightBlue] {r5} {r6} ;
\end{tikzpicture}
%
 \hspace{4cm} 
 %
\begin{tikzpicture}
\node[obs, fill=SkyBlue] (r1) {$v_1$};
\node[obs, below right=of r1, fill=SkyBlue] (r2) {$v_2$};
\node[obs, below right=of r2,  fill=SkyBlue] (r3) {$v_3$};
\node[obs, below right=of r3,  fill=SkyBlue] (r4) {$v_4$};
\edge [-, color=DarkRed] {r1} {r2} ;
\edge [-, color=DarkRed, bend right=30] {r1} {r3} ;
\edge [-, color=DarkRed] {r2} {r3} ;
%\edge [-, color=DarkRed, bend left=30] {r2} {r4} ;
\edge [-, color=DarkRed] {r3} {r4} ;
\end{tikzpicture}
\vspace{.5cm}
\caption{Examples of simple undirected graphs. In \textsc{graph A} each node has two neighbors. In \textsc{graph B} node $v_4$ has only a single neighbor while the others each have two neighbors.}
\label{fig:undirected_graphs}
\end{figure}

 An undirected graph $\mathbf{G} = (V,E)$ is simply an ordered pair of sets containing the vertices (also called nodes) and edges of the graph, respectively. In what follows assume that $V$ and $E$ are finite and let $e_{ij} \in E$ denote the element in $E$ that corresponds to the edge connecting the vertices $v_i$ and $v_j$ in $V$. The term {\it undirected} refers to the fact that the edges lack orientation.\footnote{Intuitively, it is helpful to think of the edges of an undirected graph as line segments with no implied direction, whereas the edges in a directed graph can be thought of as arrows.} The {\it neighbors} of node $v_j$ are all nodes $v_i$ such that, for all $i \neq j$ , there is an edge $e_{ij}$. The notation $\partial^{\mathbf{G}}_j$ will be used for the {\it neighborhood} of node $v_j$ in graph  $\mathbf{G}$.\footnote{$ \partial^\mathbf{G}_j = \{v_{i \neq j} \in V | e_{ij} \in E\}$} For example, in Figure~\ref{fig:undirected_graphs}, $\partial^{\mathbf{A}}_1 = \{v_2, v_6\}$ and $\partial^{\mathbf{B}}_1 = \{v_2, v_3\}$.  For a more comprehensive discussion of undirected graphs in the context of GMRFs see \citeA{rue_gaussian_2005}. 
 

\subsection{GMRFs}
Let $\theta \in \mathbb{R}^D$ have a multivariate normal distribution with probability density function $p(\theta | \mu, \boldsymbol{\Sigma}) = \mathcal{N}_D (\theta | \mu, \boldsymbol{\Sigma})$. The random vector $\theta$ is said to form a GMRF with respect to the graph $\mathbf{G} = (V,E)$ if for each element in $\theta$ there is a corresponding node in $V$, and there is no edge $e_{ij}$ between nodes $v_i$ and $v_j$ \emph{if and only if} $\theta_i$ and $\theta_j$ are conditionally independent given all other elements of $\theta$ \shortcite{rue_gaussian_2005}.\footnote{$v_i \notin \partial^\mathbf{G}_{j} \iff \theta_i \bot \theta_j \mid \theta_{-ij}$}\footnote{This definition is consistent with the definition of Markov random fields in general. The random variables $\theta_1, \dots, \theta_D$ obey the requisite Markov properties describing pairwise, local, and global conditional independencies.} 

The relationship between $G$ and $\theta$ -- the information they provide about each other -- is fully contained in the covariance matrix $\boldsymbol{\Sigma}$, but it is not obvious from the individual elements of $\boldsymbol{\Sigma}$.\footnote{Nothing about the conditional independencies among the elements of $\theta$ can be inferred from the mean vector $\mu$.} It is more useful to instead consider the precision matrix $\boldsymbol{Q}=\boldsymbol{\Sigma}^{-1}$, for which it can be shown that conditional independence between $\theta_i$ and $\theta_j$ (no edge between nodes $v_i$ and $v_j$) always corresponds to a zero in cell $ij$ of the precision matrix, and vice-versa $(\forall i \neq j, \:\: v_i \in \partial_j \iff q_{ij} = 0)$.  For a simple proof see \citeA{rue_gaussian_2005}.\footnote{For intuition, the following interpretations of the elements of a precision matrix may also be helpful to keep in mind. For $i \neq j$ (the off-diagonal), $q_{ij}  = -Cov(\theta_i, \theta_j | \theta_{-ij}) $, the negative (flipped sign) covariance between $\theta_i$ and $\theta_j$ conditional on all $\theta$s except $i$ and $j$.  For $i = j$ (the diagonal),  $q_{ij} = Var(\theta_i | \theta_{-i})$, the variance of $\theta_i$ conditional all the variables except $\theta_i$.}






\section{Bayesian STAR models}
\label{star}

An extension of generalized linear models (GLMs), semiparametric structured additive regression (STAR) models replace the linear predictor with a structured additive predictor of the form
%
\begin{equation*}
\eta = f_1(x_1) + \ldots + f_j(x_j) + \ldots + f_J(x_J) + u'\gamma  = u'\gamma + \sum_{j} f_j (x_j),
\end{equation*}
%
\noindent which can include nonlinear unknown functions of covariates as well as linear components \shortcite{fahrmeir_bayesian_2001}. 

From the Bayesian perspective, the non-varying parameters $\gamma$ as well as the unknown functions $f_1, \dots, f_J$ are treated as random variables distributed according to priors that we must specify. For an unknown function $f_j$ of a covariate $x_j$ let $\mathbf{f}_j^{eval}$ denote the vector of function evaluations of $f_j$ at each of the $N$ observed values of $x_j$ 

\begin{equation*}
\mathbf{f}_j^{eval} = \left(f_j(x_{1j}), \dots f_j(x_{Nj})\right)'.  
 \end{equation*}
 
\noindent Then $\mathbf{f}_j^{eval}$ is treated as a random vector, which, following \citeA{brezger_generalized_2006}, can conveniently be expressed as the matrix product $\mathbf{M}_j \theta_j$ of a design matrix $\mathbf{M}_j$ and a parameter vector $\theta_j$. GMRF priors for each parameter vector $\theta_j$ take the general form

\begin{equation*}
p(\theta_j | \tau^2_j) 
\propto 
\frac{1}{\left(\tau^2_j\right)^{{\rm rank}(\mathbf{P})/2} }
\exp{\left\{-\frac{1}{2\tau^2_j} \theta_j' \mathbf{P} \theta_j\right\}}
\end{equation*}

\noindent where $\mathbf{P}$ is a penalty matrix whereby we operationalize our prior assumptions about the smoothness of the unknown function $f_j$. 

To provide some intuition, suppose we are interested in learning about an effect with assumed variation over geographic regions.  A simple map with $R$ distinct regions $r_1, r_2, \dots r_R$ can be represented as undirected graph $G$ with vertices $V = \{1, 2, \dots, R\}$ and edges $E$ connecting vertices corresponding to neighboring (or proximate) regions. Assuming some degree of dependence of the quantity of interest across regions, a prior on a parameter vector $\theta = (\theta_1, \dots, \theta_r)$ can be constructed such that the matrix $\mathbf{P}$ has a zero in the $ij$th cell if and only if $\theta_i$ and $\theta_j$ (the parameters corresponding to regions $r_i$ and $r_j$) are assumed to be independent conditional on $\theta_{-ij}$. Then $\theta$ is a GMRF with respect to $G$ with $\boldsymbol{\Sigma}^{-1} = \mathbf{Q} = \mathbf{P}/\tau^2$. 

\subsection{The penalty matrix} 
\label{penalty_matrix}

More specifically, the penalty matrix $\mathbf{P}$ can be constructed as  $\mathbf{P} = \mathbf{D} - \mathbf{A}$, where $\mathbf{A}$ is a symmetric matrix with $a_{ij} = 1$ if temporal or spatial units $i$ and $j$ are considered neighbors (the associated graph $G$ has an edge between nodes $i$ and $j$) and 0 otherwise, and $\mathbf{D}$ is a diagonal matrix such that $\forall i = j, \: d_{ij} = \sum_j a_{ij}$. The matrices $\mathbf{A}$ and $\mathbf{D}$ are commonly referred to as the adjacency and degree matrices because encoded in $\mathbf{A}$ are all neighbor relationships (temporal or spatial) and the (diagonal) elements of $\mathbf{D}$ are equal to the number of neighbors (the degree) of each vertex in the graph $G$. 

To illustrate why this form of $\mathbf{P}$ captures these particular assumptions, consider $N$ measurements of a variable $x$, with each measurement of $x$ made at one of $T$ evenly spaced points in time. For simplicity, but without loss of generality,  assume that $x$ is a unit of time and there is exactly one measurement per time period.  The sequence $(x^{[t]})_{t=1}^T$ then corresponds to a grid of points on a line.  \citeA{fahrmeir_bayesian_2001} suggest several possible choices for a prior on a smooth function $f(x)$, the simplest of which is a first order random walk ($RW_1$) prior.  Under the $RW_1$ prior, the first differences $\Delta_t = f(x^{[t]}) - f(x^{[t-1]})$ are treated as independent and identically distributed standard normal random variables. 

While this formulation of the $RW_1$ prior is {\it directed}, conditioning also on $f(x^{[t+1]})$ -- one step into the future -- forms an undirected $RW_1$, where the neighbors of time $t$ are both $t-1$ and $t+1$.  The associated graph $G$ therefore has vertices $V=\{v_t : t=1,\dots,T\}$, each of which has two neighbors, with the exception of $v_1$ and $v_T$, which have one neighbor. The penalty matrix $\mathbf{P}$ corresponding to the $RW_1$ prior with equally spaced observations is the tridiagonal matrix

\begin{equation*}
\mathbf{P} = 
\begin{bmatrix}
1  	& -1 	& 		& 	& \\
-1  	& 2 	& -1 		& 	& \\
  	& -1 	& \ddots 	& \ddots	& \\
  	&  	& \ddots 	& 2 	& -1\\
  	&  	& 		& -1 	& 1\\
\end{bmatrix}
\end{equation*}

\noindent which can be derived by computing the difference of the appropriate degree and adjacency matrices \shortcite{brezger_generalized_2006}.\footnote{The construction of $\mathbf{P}$ as the matrix difference $\mathbf{D} - \mathbf{A}$ also allows for the estimation of a parameter $\omega \in [0,1]$, which, as a coefficient on $\mathbf{A}$, can be interpreted as representing the strength of spatial or temporal dependence \shortcite{rue_gaussian_2005}. The resulting precision matrix $\mathbf{Q}= (\mathbf{D} - \omega \mathbf{A})/\tau^2$ is the defining feature of the conditional autoregressive (CAR) model.  When $\omega = 1$, and thus $\mathbf{P}= \mathbf{D} - \mathbf{A}$, the model is sometimes referred to as an intrinsic CAR model.
} The $RW_1$  can be extended to an $RW_2$ prior by also considering the measurements $x^{[t-2]}$  and $x^{[t+2]}$ to be neighbors of $x^{[t]}$. 


\subsection{Hyperpriors}
\label{hyperpriors}

To complete the hierarchical model requires also specifying hyperpriors $p(\tau_j^2)$.  Assigning a prior distribution to the variance hyperparameter $\tau_j^2$ allows for the simultaneous estimation of a smoothing function $f_j$ and the amount of smoothness. \citeA{fahrmeir_bayesian_2001} and \citeA{brezger_bayesx:_2005} recommend a weakly informative but proper inverse-gamma prior on $\tau_j^2$. However, in light of concerns about this type of inverse-gamma prior raised by \citeA{gelman_prior_2006}, in this thesis several priors for $\tau^2_j$ are used and compared to check the degree to which the results are sensitive to the choice of prior.

