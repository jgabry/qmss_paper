\chapter{Discussion}
\label{discussion}

The criticism that standard quantitative methods in historical political science, and quantitative social sciences more broadly,  are unsuited for capturing essential qualitative features of history -- context, specificity, temporality, and periodicity -- should not be taken lightly. \citeA{wawro_designing_2014} argue that failure to embrace more appropriate quantitative approaches will only reinforce the existing divisions between the qualitative and quantitative researchers who otherwise share common interests and goals. 

The analyses conducted by Katznelson and Wawro along with the reanalysis of \citeA{cox_gerrymandering_2007} in this thesis provide promising case studies and show how hierarchical models with smoothness priors can better incorporate context and temporality into a quantitative historical analysis. Stan provides a flexible language for expressing these models and enables full Bayesian inference to be carried out more efficiently than previously possible.  

As the list of successful applications of Wawro and Katznelson's proposals grows, the burden will shift to the historians and qualitatively-inclined researchers to either embrace the progress made by quantitative researchers or provide statistically-informed critiques of the methods. 

  