\chapter{Conclusion}
\label{discussion}

A statistical model is simpler than the reality it seeks to describe. The process of 
distilling the features of some phenomena and representing them in the language of 
probability theory also inevitably introduces error. A statistical model is therefore both 
a reduction and expansion of reality. For this reason, social scientific applications of 
probabilistic modeling are unavailing in the absence of thoughtful qualitative work which
servers to endow the models with substantive meaning. Successful applications of quantitative 
methods are motivated by theory and supported by strong qualitative foundations. 

Along with \citeA{wawro_designing_2014}, this thesis describes and demonstrates 
an alternative to the standard statistical methods adopted for the study of historical 
political institutions and historical social scientific inquiry more broadly. The new perspective is 
motivated by the criticism from qualitatively-inclined scholars of history that traditional methods 
are unsuited for capturing essential qualitative features of history. 
Wawro and Katznelson argue that failure to embrace more thoughtful, nuanced quantitative 
approaches will only reinforce the existing divisions between qualitative and quantitative 
researchers who otherwise share common aims. 

The reanalysis of \citeA{cox_gerrymandering_2007} in Part~\ref{cox_katz} of this thesis 
provides another case study -- in addition to the analyses conducted by Katznelson and 
Wawro -- showing how hierarchical models with smoothness priors can better incorporate 
context and temporality into a quantitative historical analysis. Stan provides a flexible language 
for expressing these models and enables full Bayesian inference to be carried out 
more efficiently than previously possible.  

While successful quantitative analysis has a qualitative backbone, the converse
is often not true. The historical social scientific literature is rife with 
theories lacking corroboration (or opposition) in the form of rigorous 
statistical data analysis. In isolation, quantitative methods are nonsensical and
qualitative methods are insufficient. Their union offers the potential for the richest 
scholarship.  As the inventory of promising applications of Wawro and 
Katznelson's proposals grows, the burden will shift back to the historians and 
qualitatively-oriented scholars to either embrace the progress in quantitative analysis 
or provide informed critiques supported by statistical theory. 

