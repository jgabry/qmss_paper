\chapter{Discussion}
\label{discussion}

\section{Future directions}

For the reanalysis of \citeA{cox_gerrymandering_2007}, the hierarchical Bayesian model with first order 
undirected random walk priors is a promising start. In other work, we are exploring the sensitivity of the 
results to different possible smoothness priors including second order random walks (see \ref{reanalysis}), 
Gaussian process priors, and others. This will enable model comparisons that can 
illuminate the unique properties of the different approaches. We are also researching 
how the principles espoused by \citeA{wawro_designing_2014} can be woven into other statistical 
methods (from duration models to the detection of path dependence) and applied to diverse areas
of scholarship like democratic peace theory and studies of economic inequality. 
 

\section{Conclusion}

A statistical model is simpler than the reality it seeks to describe. In addition, the 
process of distilling the features of some phenomena into relationships concisely 
described by probability theory inevitably introduces error. A statistical model is 
therefore both a reduction and expansion of reality. For this reason, a social scientific 
application of statistical modeling is vacant in the absence of thoughtful qualitative 
work to endow the model with its meaningful information content. Successful 
applications are motivated by theory and supported by qualitative foundations. 

Following \citeA{wawro_designing_2014}, this thesis describes and demonstrates 
an alternative to the standard statistical methods adopted for the study of historical 
political institutions and, more broadly, historical social scientific inquiry. The new perspective is 
motivated by the criticism from qualitatively-inclined scholars of history that traditional methods 
are unsuited for capturing essential qualitative features of history. 
Wawro and Katznelson argue that failure to embrace more thoughtful, nuanced quantitative 
approaches will only reinforce the existing divisions between qualitative and quantitative 
researchers who otherwise share common aims. 

The reanalysis of \citeA{cox_gerrymandering_2007}
provides another case study -- in addition to the analyses conducted by Katznelson and 
Wawro -- showing how hierarchical models with smoothness priors can better incorporate 
context and temporality into a quantitative historical analysis. These models are 
natural to construct in a Bayesian framework, accommodate greater parameter variation, 
and allow periodization schemes be substantially data-dependent.  
Stan provides a flexible language for expressing these models and 
enables full Bayesian inference to be carried out more efficiently than 
previously possible.  

While successful quantitative analysis has a qualitative backbone, 
the converse is often not true. The historical social scientific literature 
is rife with theories lacking corroboration (or opposition) in the form of rigorous 
data analysis. In isolation, quantitative methods are hollow and
qualitative methods are insufficient. Their union offers the potential for 
the richest scholarship. As the inventory of promising applications of Wawro and 
Katznelson's proposals grows, hopefully historians and qualitatively-oriented 
scholars will embrace the progress in quantitative analysis and provide informed 
critiques supported by statistical theory. 

