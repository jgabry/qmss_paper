\chapter{Conclusion}
\label{discussion}

The criticism that standard quantitative methods applied to the study of historical political institutions -- and historical social scientific inquiry broadly -- are unsuited for capturing essential qualitative features of history should not be taken lightly. \citeA{wawro_designing_2014} argue that failure to embrace more appropriate quantitative approaches will only reinforce the existing divisions between qualitative and quantitative researchers who otherwise share common interests and goals. 

The reanalysis of \citeA{cox_gerrymandering_2007} in this thesis provides another case study -- in addition to the analyses conducted by Katznelson and Wawro -- showing how hierarchical models with smoothness priors can better incorporate context and temporality into a quantitative historical analysis. Stan provides a flexible language for expressing these models and enables full Bayesian inference to be carried out more efficiently than previously possible.  

As the list of successful applications of Wawro and Katznelson's proposals grows, the burden will shift to the historians and qualitatively-inclined researchers to either embrace the quantitative progress made or provide statistically-informed critiques of the methods. 

  