\chapter{Farhang and Katznelson}

To illustrate the utility of STAR models, Wawro and Katznelson (2013) conduct a reanalysis of Farhang and Katznelson (2005). Farhang and Katznelson argue that drastic changes in the Democratic party's position on labor-friendly policies occurred during the time period of time between New Deal and Fair Deal, largely due to a growing divide between norther and southern Democrats. 

In this section we expand on the reanalysis proposed by Wawro and Katznelson. 

% Something about how this analysis includes time and space as opposed to cox and katz which is just time

\section{Data} 

The data used by Farhang and Katznelson pertains to roll-call votes in the 73rd through 80th Congresses (1933--1948). The outcome of interest is a binary variable indicating whether a senator $s$ from region $r$ voted in favor of the prolabor position on roll-call vote $v$ in Congress $t$

$$L_{srtv} = 
\begin{cases}
1, & \text{ if senator $s$ from region $r$ votes prolabor on vote $v$ at time $t$} \\
0, & \text{ otherwise.}
\end{cases}
$$

The possible values for region are Non-South (NS), Border South (BS), and Deep South (DS). 

Predictors include the proportion of African Americans living in the senator's home region $(AA)$, the proportion of individual's in the senator's home region living in urban areas ($URB$), a measure of unionization 
%\footnote{Include description from W\&K paper}
$(UN)$, and indicators for Democratic party membership $(DEM)$ and membership on the committee charged with overseeing labor-related issues $(COM)$.  % this is maybe too close to the wording in W&K

% say something about why these variables are relevant

% FIGURES summarizing data % 

\section{Method of analysis}

\subsection{The map as an undirected graph}

Unlike in the Cox and Katz analysis, here we want to account for variation and dependence over both temporal and spatial dimensions. From the three regions and eight congresses in the data we construct the $3 \times 8$ map with cells $c_1,c_2, \dots, c_{24}$ shown in Table~\ref{table:map_degree}. The value displayed in cell $c_{j}$ is the number of cells considered neighbors of $c_{j}$

$$ c_{j} = \sum_{k = 1}^{24} \textit{Ne}(j,k),$$

\noindent where the neighbor function $\textit{Ne}(j,k)$ is 1 if $c_j$ and $c_k$ are neighbors and $0$ otherwise. The value of each $c_j$ is determined by our subjective choice of the form of the neighbor function, which should be informed by a substantive theory about the temporal and spatial dependence. 

The values in Table~\ref{table:map_degree} correspond to using an undirected neighbor function that considers adjacency to be vertical, horizontal and diagonal. For example, this form of the neighbor function allows for dependence between the Border South region in the 76th Congress and the Non-South region in the 77th Congress. Other possible neighbor functions might only allow for dependence across regions in the same time period, or only allow dependence on prior time periods, or allow dependence on time periods even further into the past or future.    

%\begin{table}[ht]
%\centering
%\begin{tabular}{l|rrrrrrrr}
%\textbf{Region}/\textbf{Congress} & 73rd & 74th & 75th & 76th & 77th & 78th & 79th & 80th \\ 
%  \toprule
%Non-South &   1 &   4 &   7 &  10 &  13 &  16 &  19 &  22 \\ 
%  Border South &   2 &   5 &   8 &  11 &  14 &  17 &  20 &  23 \\ 
%  Deep South &   3 &   6 &   9 &  12 &  15 &  18 &  21 &  24 \\ 
%   \bottomrule
%\end{tabular}
%\caption{\small The $3 \times 8$ map assigning a unique identifier for each region/Congress pair}
%\label{table:map_id}
%\end{table}


\begin{table}[ht]
\centering
\begin{tabular}{l|rrrrrrrr}
\textbf{Region}/\textbf{Congress} & 73rd & 74th & 75th & 76th & 77th & 78th & 79th & 80th \\ 
 \toprule
Non-South 	
& $3_{(1)}$ & $5_{(4)}$ & $5_{(7)}$ & $5_{(10)}$ & $5_{(13)}$ & $5_{(16)}$ & $5_{(19)}$ & $3_{(22)}$ \\ 
Border South 	
& $5_{(2)}$ & $8_{(5)}$ & $8_{(8)}$ & $8_{(11)}$ & $8_{(14)}$ & $8_{(17)}$ & $8_{(20)}$ & $5_{(23)}$ \\ 
Deep South 	
& $3_{(3)}$ & $5_{(6)}$ & $5_{(9)}$ & $5_{(12)}$ & $5_{(15)}$ & $5_{(18)}$ & $5_{(21)}$ & $3_{(24)}$ \\ 
   \bottomrule
\end{tabular}
\caption{\small The $3 \times 8$ map where the cell contents are the number of neighbors. The subscripts assign each region/Congress pair a unique identifier from 1 to 24. Adjacency is vertical, horizontal, and diagonal.}
\label{table:map_degree}
\end{table}


Using the indices $r \in \{1,\dots, R\}$ (where in this case $R = 24$) from Table~\ref{table:map_degree}, we can 
reconceptualize the map as an undirected graph with $R$ vertices and with edges connecting neighbors. We can then make the two $R \times R$ matrices -- the symmetric adjacency matrix and diagonal degree matrix -- needed for constructing the precision matrix for the GMRF prior. 


%\clearpage
%
%\subsection{Notation}
%
%\begin{tabular}{ll}
%$N \approx 8000$ & Number of observations in the data. \\[5pt]
%
%$R = 24$ & Number of groups (i.e. region/period combos).  \\[5pt]
%
%$\textsc{urb}$, $\textsc{aa}$, $\textsc{un}$ & The three $N \times 1$ vectors for {\tt urbanpct}, {\tt aapct}, and {\tt unionpop} variables. \\[5pt]
%
%$\mathbf{M}$ & $N \times R$ matrix with $m_{ij} = 1$ if obs $i$ is from region-period $j$, and 0 otherwise. \\[5pt]
%
%$\mathbf{X}_1, \mathbf{X}_2, \mathbf{X}_3$ & $N \times R$ matrices with $x_{ij} =$ $i$th value of 
% \textsc{urb} (for $\mathbf{X}_1$), \textsc{aa} (for $\mathbf{X}_2$), \\
% 
% & or \textsc{un} (for $\mathbf{X}_3$) if obs $i$ is from region-period $j$, and 0 otherwise. \\[5pt]
%
%$\mathbf{Z} $ & $N \times 2$ matrix with party and committee indicators.  \\[5pt]
%
%$y$ & $N \times 1$ vector containing the indicator for pro-labor vote. 
%
%\end{tabular}



\vskip1in
\subsection{Model}

Conditional on parameters $\theta$, the observations of the binary indicator $L$ are assumed to be independent and follow a Bernoulli distribution. Here, the structured additive predictor $\eta = {\rm logit}(\theta)$ is 

$$\log{\left(\frac{\theta_{srtv}}{1 - \theta_{srtv}}\right)} = \eta_{srvt} = \alpha_0 + f_1 (\alpha_{rt}) + f_2 (UN_{srt}) + f_3 (URB_{srt}) + f4(AA_{srt}) + u_{srt}' \gamma, $$

\noindent where $\alpha_0$ represents a global intercept, $\alpha_{rt}$ is the region-period specific deviation from $\alpha_0$, and $\mathbf{u}$ includes $DEM$ and $COM$, predictors with effects assumed not to vary across region and time period.
 
As in the Cox and Katz model, we can express the vector of evaluations of each unknown function $f_j$ as the product of a design matrix $\mathbf{M}_j$ and a parameter vector, here denoted $\beta_j$. We can then write the structured additive predictor $\eta$ in matrix form as 
 
$$\eta = \mathbf{u}\gamma + \alpha_0 + \left( \sum_{j=1}^{4} \mathbf{M}_j \beta_j \right). $$
 
Specifying GMRF smoothness priors over regions and periods corresponds to the zero-mean multivariate normal priors

$$ p(\beta_j | \tau^2_j) \propto  \exp{\left(-\frac{1}{2\tau_j^2} \beta_j^T \mathbf{P}^{-1} \beta_j \right)}. $$

Weakly informative Gaussian priors are used for the coefficients $\gamma$. Prior distributions for the global intercept $\alpha_0$ and the hyperparameters $\tau_j$ are discussed below in the Estimation section. 








%For $j = 1, \dots, J$ the precision matrix $\mathbf{K}_j$ is defined as $\mathbf{K}_j = \tau_j \left(\mathbf{D} - \rho_j \mathbf{A}\right).$ The matrices $\mathbf{A}$ and $\mathbf{D}$ are the symmetric adjacency matrix and diagonal degree matrix. 
%
%\vskip 1cm
%
%For each precision matrix $\mathbf{K}_{R \times R}$, we can interpret its elements as follows:
%
%\begin{itemize}
%\item For $i \neq j$, element $k_{ij}$ is conditional covariance between $i$ and $j$, given all the variables except $i$ and $j$, except the sign of the conditional covariance is flipped.
%
%\item For $i = j$, element $K_{ij}$ is the conditional variance of $i$, given all the variables except $i$.
%\end{itemize}
%
%This enables us to go from the full-conditional relationships that GeoBUGS utilizes for Gibbs sampling to the joint form that Stan needs for HMC. \\
%
%


%\subsection{Hyperpriors}
%
%Let $\lambda_j \sim {\rm Exponential}(1)$ and let $\tau_j = \bar{\tau}\lambda_j $, which implies
%
%$$\tau_j \sim {\rm Exponential}(1/\bar{\tau}), $$
%
%where $\bar{\tau} > 0$ is the common mean for the $\tau_j$'s. Give $\bar{\tau}$ an improper flat prior over $\mathbb{R}^+.$ \\
%
%
%
%
%\noindent Let $\rho_j \sim {\rm Beta}(s_1, s_2)$, where $s_1 = \gamma \bar{\rho}$,  and $s_2 = \gamma(1-\bar{\rho})$, where $\gamma$ is either an estimated positive dispersion parameter or fixed (e.g. $\gamma = 1$). For the common mean $\bar{\rho} \in (0,1)$ let $\bar{\rho} \sim {\rm Unif}(0,1)$.    \\







%
%\begin{table}[ht]
%\scriptsize
%\centering
%$$  \begin{array}{r|rrrrrrrrrrrrrrrrrrrrrrrr} 
%  & 1 & 2 & 3 & 4 & 5 & 6 & 7 & 8 & 9 & 10 & 11 & 12 & 13 & 14 & 15 & 16 & 17 & 18 & 19 & 20 & 21 & 22 & 23 & 24 \\ 
%\midrule
%1 & 3 & \!-1\! & 0 & \!-1\! & \!-1\! & 0 & 0 & 0 & 0 & 0 & 0 & 0 & 0 & 0 & 0 & 0 & 0 & 0 & 0 & 0 & 0 & 0 & 0 & 0 \\ 
%  2 & \!-1\! & 5 & \!-1\! & \!-1\! & \!-1\! & \!-1\! & 0 & 0 & 0 & 0 & 0 & 0 & 0 & 0 & 0 & 0 & 0 & 0 & 0 & 0 & 0 & 0 & 0 & 0 \\ 
%  3 & 0 & \!-1\! & 3 & 0 & \!-1\! & \!-1\! & 0 & 0 & 0 & 0 & 0 & 0 & 0 & 0 & 0 & 0 & 0 & 0 & 0 & 0 & 0 & 0 & 0 & 0 \\ 
%  4 & \!-1\! & \!-1\! & 0 & 5 & \!-1\! & 0 & \!-1\! & \!-1\! & 0 & 0 & 0 & 0 & 0 & 0 & 0 & 0 & 0 & 0 & 0 & 0 & 0 & 0 & 0 & 0 \\ 
%  5 & \!-1\! & \!-1\! & \!-1\! & \!-1\! & 8 & \!-1\! & \!-1\! & \!-1\! & \!-1\! & 0 & 0 & 0 & 0 & 0 & 0 & 0 & 0 & 0 & 0 & 0 & 0 & 0 & 0 & 0 \\ 
%  6 & 0 & \!-1\! & \!-1\! & 0 & \!-1\! & 5 & 0 & \!-1\! & \!-1\! & 0 & 0 & 0 & 0 & 0 & 0 & 0 & 0 & 0 & 0 & 0 & 0 & 0 & 0 & 0 \\ 
%  7 & 0 & 0 & 0 & \!-1\! & \!-1\! & 0 & 5 & \!-1\! & 0 & \!-1\! & \!-1\! & 0 & 0 & 0 & 0 & 0 & 0 & 0 & 0 & 0 & 0 & 0 & 0 & 0 \\ 
%  8 & 0 & 0 & 0 & \!-1\! & \!-1\! & \!-1\! & \!-1\! & 8 & \!-1\! & \!-1\! & \!-1\! & \!-1\! & 0 & 0 & 0 & 0 & 0 & 0 & 0 & 0 & 0 & 0 & 0 & 0 \\ 
%  9 & 0 & 0 & 0 & 0 & \!-1\! & \!-1\! & 0 & \!-1\! & 5 & 0 & \!-1\! & \!-1\! & 0 & 0 & 0 & 0 & 0 & 0 & 0 & 0 & 0 & 0 & 0 & 0 \\ 
%  10 & 0 & 0 & 0 & 0 & 0 & 0 & \!-1\! & \!-1\! & 0 & 5 & \!-1\! & 0 & \!-1\! & \!-1\! & 0 & 0 & 0 & 0 & 0 & 0 & 0 & 0 & 0 & 0 \\ 
%  11 & 0 & 0 & 0 & 0 & 0 & 0 & \!-1\! & \!-1\! & \!-1\! & \!-1\! & 8 & \!-1\! & \!-1\! & \!-1\! & \!-1\! & 0 & 0 & 0 & 0 & 0 & 0 & 0 & 0 & 0 \\ 
%  12 & 0 & 0 & 0 & 0 & 0 & 0 & 0 & \!-1\! & \!-1\! & 0 & \!-1\! & 5 & 0 & \!-1\! & \!-1\! & 0 & 0 & 0 & 0 & 0 & 0 & 0 & 0 & 0 \\ 
%  13 & 0 & 0 & 0 & 0 & 0 & 0 & 0 & 0 & 0 & \!-1\! & \!-1\! & 0 & 5 & \!-1\! & 0 & \!-1\! & \!-1\! & 0 & 0 & 0 & 0 & 0 & 0 & 0 \\ 
%  14 & 0 & 0 & 0 & 0 & 0 & 0 & 0 & 0 & 0 & \!-1\! & \!-1\! & \!-1\! & \!-1\! & 8 & \!-1\! & \!-1\! & \!-1\! & \!-1\! & 0 & 0 & 0 & 0 & 0 & 0 \\ 
%  15 & 0 & 0 & 0 & 0 & 0 & 0 & 0 & 0 & 0 & 0 & \!-1\! & \!-1\! & 0 & \!-1\! & 5 & 0 & \!-1\! & \!-1\! & 0 & 0 & 0 & 0 & 0 & 0 \\ 
%  16 & 0 & 0 & 0 & 0 & 0 & 0 & 0 & 0 & 0 & 0 & 0 & 0 & \!-1\! & \!-1\! & 0 & 5 & \!-1\! & 0 & \!-1\! & \!-1\! & 0 & 0 & 0 & 0 \\ 
%  17 & 0 & 0 & 0 & 0 & 0 & 0 & 0 & 0 & 0 & 0 & 0 & 0 & \!-1\! & \!-1\! & \!-1\! & \!-1\! & 8 & \!-1\! & \!-1\! & \!-1\! & \!-1\! & 0 & 0 & 0 \\ 
%  18 & 0 & 0 & 0 & 0 & 0 & 0 & 0 & 0 & 0 & 0 & 0 & 0 & 0 & \!-1\! & \!-1\! & 0 & \!-1\! & 5 & 0 & \!-1\! & \!-1\! & 0 & 0 & 0 \\ 
%  19 & 0 & 0 & 0 & 0 & 0 & 0 & 0 & 0 & 0 & 0 & 0 & 0 & 0 & 0 & 0 & \!-1\! & \!-1\! & 0 & 5 & \!-1\! & 0 & \!-1\! & \!-1\! & 0 \\ 
%  20 & 0 & 0 & 0 & 0 & 0 & 0 & 0 & 0 & 0 & 0 & 0 & 0 & 0 & 0 & 0 & \!-1\! & \!-1\! & \!-1\! & \!-1\! & 8 & \!-1\! & \!-1\! & \!-1\! & \!-1\! \\ 
%  21 & 0 & 0 & 0 & 0 & 0 & 0 & 0 & 0 & 0 & 0 & 0 & 0 & 0 & 0 & 0 & 0 & \!-1\! & \!-1\! & 0 & \!-1\! & 5 & 0 & \!-1\! & \!-1\! \\ 
%  22 & 0 & 0 & 0 & 0 & 0 & 0 & 0 & 0 & 0 & 0 & 0 & 0 & 0 & 0 & 0 & 0 & 0 & 0 & \!-1\! & \!-1\! & 0 & 3 & \!-1\! & 0 \\ 
%  23 & 0 & 0 & 0 & 0 & 0 & 0 & 0 & 0 & 0 & 0 & 0 & 0 & 0 & 0 & 0 & 0 & 0 & 0 & \!-1\! & \!-1\! & \!-1\! & \!-1\! & 5 & \!-1\! \\ 
%  24 & 0 & 0 & 0 & 0 & 0 & 0 & 0 & 0 & 0 & 0 & 0 & 0 & 0 & 0 & 0 & 0 & 0 & 0 & 0 & \!-1\! & \!-1\! & 0 & \!-1\! & 3 \\ 
%\end{array} $$
%\caption{The ``Laplacian matrix" $\mathbf{L} = \mathbf{D} - \mathbf{A}$. The precision matrix $\mathbf{K}_j$ is $\mathbf{K}_j = \tau_j (\mathbf{D} - \rho_j \mathbf{A})$, where $\tau_j  > 0$ and $\rho_j \in (0,1)$. Thus $\rho_j \to 1 \implies \mathbf{K}_j \to \tau_j \mathbf{L}$.}
%\end{table}



