\clearpage
\label{AppendixD}
\vspace{-1.75cm}
\subsection{Hamiltonian Monte Carlo}

\noindent The following is a simplified version of the HMC algorithm employed by Stan. 

{\singlespacing
\subsubsection{Notation}

\begin{tabular}{ll}
$\theta$ & $D$-dimensional position vector (model parameters to be estimated) \\

 $ \rho$ & $D$-dimensional vector of auxiliary momentum variables \\

$\mathcal{L}$ & Natural logarithm of joint probability density for $\theta$ (up to normalizing constant) \\

$\nabla_\theta$ & Gradient with respect to $\theta$ \\

$x^{[t]}$ & Value of $x$ at time $t$ \\

$N$ & Number of iterations to run the algorithm \\

$L$ & Number of ``leapfrog" steps (to run the simulated Hamiltonian system) per iteration\\

$\epsilon$ & Size of the leapfrog steps
\end{tabular}
}


\subsubsection{Leapfrog function}

\noindent A single leapfrog update consists of executing the following function once:

{\singlespacing

${\tt leapfrog} (\rho, \theta, \epsilon) \{$ 

\begin{tabular}{ll}
$\quad \tilde{\rho} \leftarrow \rho + \tfrac{1}{2}\epsilon \nabla_\theta \mathcal{L}(\theta)$ & {\it (half step)} \\

$\quad \tilde{\theta} \leftarrow \theta + \epsilon \tilde{\rho}$ & {\it (full step)} \\

$\quad \tilde{\rho} \leftarrow \tilde{\rho} + \tfrac{1}{2}\epsilon \nabla_\theta \mathcal{L}(\tilde{\theta})$ & {\it (half step)} \\

$\quad {\tt return} (\tilde{\rho}, \tilde{\theta})$
\end{tabular}

$\}$

}

\subsubsection{Pseudo-code for HMC algorithm}


\noindent Set $N$, $\theta^{[0]}$, $\epsilon$, and $L$.

\noindent For $n = 1, \dots, N \quad \{$ 

\begin{tabular}{ll}
{\bf (i)} $\rho^{[0]} \sim \mathcal{N}_D(0, \mtrx{I})$ & ({\it Generate random momentum}) \\

{\bf (ii)} $\tilde{\rho} \leftarrow \rho^{[0]}$, $\tilde{\theta} \leftarrow \theta^{[n-1]}$ &  ({\it Initialize temporary $\tilde{\rho}, \tilde{\theta}$})\\

{\bf (iii)} For $\ell = 1, \dots, L \quad \{$  & ({\it Do $L$ leapfrog updates to generate proposal }) \\[-8pt]

\qquad $(\tilde{\rho}, \tilde{\theta}) \leftarrow {\tt leapfrog}(\tilde{\rho}, \tilde{\theta}, \epsilon)$ & \\[-8pt]
\quad $\}$ & \\

\quad Compute acceptance probability $p^\star$  & \\[-8pt]
$\qquad  q^{\it new} \leftarrow  \mathcal{L}(\tilde{\theta}) - \tfrac{1}{2} \tilde{\rho} \cdot \tilde{\rho} $ & \\[-8pt]
$ \qquad q^{\it old} \leftarrow   \mathcal{L}(\theta^{[n-1]} ) -  \tfrac{1}{2} \rho^{[0]} \cdot \rho^{[0]} $ & \\[-8pt]
$ \qquad p^\star = \min{\left\{1, \exp{\left( q^{\it new} - q^{\it old} \right)} \right\}} $& \\[3pt]

{\bf (iv)} Set $\theta^{[n]} \leftarrow \tilde{\theta}$ with probability $p^\star$  & ({\it Accept or reject proposal})\\
\qquad and $\theta^{[n]} \leftarrow \theta^{[n-1]}$ with probability $ 1 - p^\star$ & \\

\end{tabular}

\noindent $\}$

In {\bf (iv)} the proposal for $m$ is accepted or rejected along with the proposal for 
$\theta$, but in practice we only care about $\theta$. The momentum variables are 
introduced out of necessity in order to simulate Hamiltonian dynamics, but they are 
not themselves quantities of interest.

Note that the above procedure requires $\epsilon$ (step-size) and $L$ (number of steps) to be specified in 
advance. Very roughly speaking, if $L$ and/or $\epsilon$ is too small proposals for $\theta^{[n]}$ will be too 
close to $\theta^{[n-1]}$ and the chain will mix slowly, whereas if they are too large the proposals will stray 
too far and will be unlikely to be accepted. Specifically, too small values for $L$ lead to unwanted random 
walk behavior and overly large values result in superfluous computations. Larger values for $\epsilon$ will 
also lead to cruder discrete approximations of the continuous-time Hamiltonian system \shortcite{hoffman_2012}.

Stan's adaptation algorithm based on the No-U-Turn sampler (NUTS) \shortcite{hoffman_2012} 
is designed to automatically tune $L$ and $\epsilon$ for each chain during a warmup period. 
See \citeA{hoffman_2012} for the technical details and \citeA{stan_development_team_stan_2015} 
for more on Stan's implementation. 
