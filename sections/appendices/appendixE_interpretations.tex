\chapter*{Appendix E}\label{AppendixE}
\vspace{-1.75cm}
\subsection{Different Interpretations}

Although Figure~\ref{fig:ck_bias} (p.~\pageref{fig:ck_bias}) appears to show two versions of the same plot with
slightly different estimates, Cox and Katz's results -- obtained by maximum likelihood estimation --
and the results from the Bayesian model must be interpreted differently. The former are point estimates 
and confidence intervals, whereas the full Bayesian model provides an estimate of the marginal posterior 
{\it distribution} for each parameter. That is, the estimates obtained from the Bayesian model are the 
distributions themselves, which are representations of our uncertainty about the parameters 
given the data. 

The methods used by Cox and Katz come from the classical approach where a single point 
estimate is obtained for each parameter and assessed in relation to infinitely many intervals 
constructed from datasets obtained from hypothetical, endless repeated sampling. While in many 
situations it is natural to imagine repeating an experiment (e.g. coin flipping) or resampling from 
a population, such an interpretation is more difficult to justify in the context of the historical 
events of interest to this analysis. For each confidence interval obtained from Cox and Katz's analysis, 
what can be said is that it either contains the (theoretical) true parameter value or it does not. This is 
a consequence of accepting (explicitly or implicitly) the classical position that the parameter is fixed 
while the data and confidence interval bounds are random variables. 

On the contrary, from the Bayesian perspective the parameter is treated as a random variable while 
the data and interval bounds are fixed. As a result, any so-called credible intervals computed from the 
posterior distribution have a more intuitive interpretation: the probability that the true parameter value lies 
in the $p\%$ interval  -- conditional on the model and data -- is $p$. 

