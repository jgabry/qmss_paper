\chapter*{Appendix E}\label{AppendixE}
\vspace{-1.75cm}
\subsection{Different interpretations}

Cox and Katz's results obtained by maximum likelihood estimation 
and the results from the Bayesian model must be interpreted differently. The former are point estimates 
and confidence intervals, whereas the full Bayesian model provides an estimate of the marginal posterior 
{\it distribution} for each parameter. The estimates obtained from the Bayesian model are the 
distributions themselves, each of which represents an approximation of our a posteriori
uncertainty about a certain parameter. 

The methods used by Cox and Katz come from the classical approach where a single point 
estimate is computed for each parameter and assessed in relation to infinitely many potential 
intervals computed from infinitely many potential samples. %
%While in many 
%situations it is natural to imagine repeating an experiment (e.g. coin flipping) or resampling from 
%a population, such an interpretation is more difficult to reconcile with the context of the historical 
%events of interest to this analysis. 
But for an individual interval,
what can be said is that it either contains the (abstract) true parameter value or it does not. This is 
a consequence of accepting (explicitly or implicitly) the classical position that the parameter is fixed 
but the data and interval bounds are random variables. 

On the contrary, from the Bayesian perspective the parameter is treated as a random variable while 
the data and intervals are fixed. As a result, any so-called credible intervals computed from the 
posterior distribution have a more intuitive interpretation: the probability that the true parameter value lies 
in the $p\%$ interval  -- conditional on the model and data -- is $p$. 

