\chapter*{Appendix A}\label{AppendixA}
\vspace{-1.75cm}
\subsection{Improper priors}

The penalty matrix $\mathbf{P}$ corresponding to an undirected $RW_\kappa$ prior over 
parameter vector $\lambda \in \mathbb{R}^C$ has rank $C - \kappa$ and the resulting 
matrix $\Sigma^{-1} = \mathbf{Q} = \mathbf{P}/\tau^2$ is singular and thus not a valid precision 
matrix. In the $RW_1$ case this can be mitigated by jittering the elements of $\mathbf{P}$. 
For $\kappa > 1$ a possibility is to estimate only $C - \kappa$ of the $C$ parameters in $\lambda$,  
from which the remaining $\kappa$ so-called {\it pinned} parameters (each of which has a 
conditional variance of zero) can be computed following the properties of the multivariate 
normal distribution. This strategy amounts to  constraining the relevant linear combinations 
to have zero variance, from which a proper prior on a lower dimensional subspace is 
obtained \shortcite{paciorek_2009, rue_gaussian_2005}. 

Another possibility is to estimate the autoregressive parameter $\omega$ mentioned 
in \ref{penalty_matrix} (see footnote~\ref{footnote_car}), which results in a proper prior.
\citeA<See>{paciorek_2009} \citeA<and>{banerjee_hierarchical_2004} for potential
drawbacks to this approach.

An entirely different tactic would be to estimate an arbitrary tridiagonal precision matrix 
rather than specifying the penalty matrix and estimating the hierarchical variance/precision 
parameter. One way to do this would be to decompose the precision matrix into conditional 
correlations and standard deviations. This is possible to do in Stan and should be explored 
in future work to sidestep the matrix singularity problems. 





