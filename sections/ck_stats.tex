\section{Statistical and computational methods}
\label{ck_stats}

\subsection{Description of statistical model}
\label{subsection_methods}

\subsubsection{Cox and Katz's analysis}

\citeA{cox_gerrymandering_2007} propose two methods for checking their hypothesis 
of bias towards the majority party during the czar rule and post-packing eras. One strategy 
is to estimate separate models for four historical time periods of interest. 
\citeA{goodrich_designing_2012} advises that this ``imposes a particular periodization scheme 
that, while consistent with received wisdom, may or may not be the correct one" (p. 16). 

Cox and Katz's second approach is more promising in that it allows for greater parameter 
heterogeneity and makes fewer a priori assumptions about the underlying data generating 
process. It is worth noting that making assumptions about the data generating process is in 
general not only unavoidable but also extremely important. The problem with the assumptions 
required for Cox and Katz's first method is not that they are implausible, but rather that 
periodization schemes per se are convenient abstractions that, in the context of quantitative 
historical analysis, can enforce a theoretically motivated but not empirically justified structure 
that should be learned rather than imposed. However, as demonstrated below, there are also 
issues with Cox and Katz's second design that can be overcome by following the recommendations 
in \citeA{wawro_designing_2014}. 

Let $C$ denote the number of unique Congresses (time periods) in the data. 
The analysis conducted by Cox and Katz centers on estimating the parameters  
$\lambda = (\lambda_t : t = 1, \dots, C)$ representing bias (on the logit scale) toward 
the majority party in each Congress $t$. They do this by maximum likelihood estimation 
of grouped logit models with linear predictor $ \lambda_t + \rho_t \log{\left(v_t^{Ratio} \right)}$. 

The estimation of $\lambda$ and $\rho$ by grouped logit models follows naturally from 
solving the seats-votes equation for the average seat share. In the legislative context of 
Cox and Katz's example this means solving for the expected roll-call win share for the 
majority as

\begin{equation*}
  E(\pi_t)  = \left(1 + \exp{\left\{- \lambda_t - \rho_t \log{\left( v_t^{Ratio}  \right)}\right\}}\right)^{-1},
\end{equation*}
%
\noindent which is the familiar logistic function. 

Cox and Katz's strategy is to estimate what they call ``a sort of running average" of bias 
across time (p. 116). To do this they take as their estimate of $\lambda_t$ the average 
estimate of $\lambda$ over the seven congresses centered at $t$. However, this 
approach to modeling temporal dependence in $\lambda$ requires reusing the data 
to estimate models for each Congress -- the observations for each Congress are used 
up to seven times.  \citeA{goodrich_designing_2012} points out that such recycling of 
data can lead to overly precise parameter estimates (this point is discussed further and 
corroborated visually in \ref{results}). 

Although Cox and Katz do not address this potential for exaggerated precision, they do 
call attention to another  concern. To obtain reasonable estimates, their method requires 
nontrivial variation in average vote share between the seven Congresses centered at 
$t$.\footnote{Technically there will be fewer than seven Congresses for $t < 4$ as well as 
$t > C- 3$, but the idea is the same regardless of the exact number.}

\subsubsection{Alternative analysis with Bayesian STAR model}

The alternative analysis presented below overcomes both of these concerns by employing 
a hierarchical Bayesian framework which takes advantage of partial pooling. Following Cox 
and Katz, a BetaBinomial likelihood is used, however the linear predictor is replaced by the 
structured additive predictor $\eta$. 

The ${\rm Beta\text{-}Binomial}(n,\alpha,\beta)$ distribution can be thought of as a compound 
distribution resulting from a ${\rm Binomial}(n,p)$ distribution where $p \sim {\rm Beta}(\alpha,\beta)$. 
Rather than assuming a fixed probability of a majority party roll-call victory -- in which case a simple 
binomial model would suffice -- the beta-binomial distribution treats the underlying probability $p$ 
as a random variable. The parameters $\alpha > 0$ and $\beta > 0$ govern the shape of the beta 
distribution, however it is both more intuitive and computationally attractive to reparameterize in terms 
of the mean $\theta = \alpha / (\alpha + \beta)$, which requires introducing an auxiliary parameter 
$\phi = \alpha + \beta$. 

This parameterization allows for modeling ${\rm logit}(\theta)$ by the semi-parametric structured 
additive predictor $\eta$ of the STAR model, which can be written in vector/matrix notation 
as 

\begin{equation*}
 \eta = f_\lambda(t) +  f_\rho(\log{(v^{Ratio})}). 
\end{equation*}

\noindent As discussed in \ref{star}, the vectors of function evaluations are 
expressed as the products 

\begin{equation*}
\mathbf{f}^{eval}_\lambda = \mtrx{M}_\lambda \lambda, 
\qquad 
\mathbf{f}^{eval}_\rho =  \mtrx{M}_\rho \rho, 
\end{equation*}
%
\noindent of the parameter vectors $\lambda \in \mathbb{R}^C$ and $\rho \in \mathbb{R}^C$ 
premultiplied by the $N \times C$ design matrices  $\mathbf{M}_\lambda$ and  $\mathbf{M}_\rho$ 
($N$ is the number of observations). The matrices $\mathbf{M}_\lambda$ and $\mathbf{M}_\rho$ 
are identical in structure but not content. The elements of $\mathbf{M}_\lambda$ are zeros and ones 
indicating the Congress to which each observation pertains. For $\mathbf{M}_\rho$ each element is 
either a zero or the appropriate value of $\log{(v^{Ratio})}$. 

% Discuss challenges with sparse matrices? Either here or in the lit review section

Priors for $f_{\lambda}$ and $f_{\rho}$ are expressed as distributions over the random 
vectors $\lambda$ and $\rho$ as 

\begin{equation*}
p(\lambda | \tau_\lambda^2, \bar{\lambda}) 
\propto 
\exp{\left\{-\frac{1}{2\tau_\lambda^2} \: \left(\lambda - \bar{\lambda}\right)^\intercal  \mtrx{P}   
\left(\lambda - \bar{\lambda}\right) \right\}} 
\propto 
\mathcal{N} (\lambda | \bar{\lambda}, \mtrx{Q}_\lambda^{-1}), 
\end{equation*}
\begin{equation*}
p(\rho | \tau_\rho^2, \bar{\rho}) 
\propto 
\exp{\left\{-\frac{1}{2\tau_\rho^2} \: \left(\rho - \bar{\rho}\right)^\intercal  \mtrx{P} 
\left(\rho-\bar{\rho}\right) \right\}} 
\propto 
\mathcal{N} (\rho | \bar{\rho}, \mtrx{Q}_\rho^{-1}), 
\end{equation*}

\noindent where $\mathbf{P}$ is the penalty matrix.

To model temporal dependence such that the neighbors of Congress $t$ also provide 
information about the values of the parameters of interest at time $t$, the undirected 
random walk prior discussed in \ref{penalty_matrix} is used. Using the notation 
introduced in \ref{undirected_graphs} we can define adjacency by specifying the 
neighborhoods for all Congresses (time periods) $t = 1, \dots, C$

\begin{equation*}
\partial^\mathbf{G}_t = \left\{C_s : \left| s - t \right| \leq \kappa \right\},
\end{equation*}

\noindent where $\kappa \leq C$ is a positive integer and $\mathbf{G}$ is the undirected graph 
with nodes $V = \{\nu_1, \dots, \nu_C\}$ and the edges implied by $\partial^\mathbf{G}$ 
(see Figure~\ref{fig:rw_undirected_graphs}). The results presented in section~\ref{results} use 
the first order undirected random walk prior for $f_\lambda$ and $f_\rho$, which applies smoothing 
while allowing for slightly more abrupt shifts in parameter values between Congresses than the 
$RW_2$ prior.\footnote{As mentioned in \ref{hyperpriors}, several priors for $\tau^2_\lambda$ 
and $\tau^2_\rho$ were considered including inverse-gamma \shortcite{fahrmeir_bayesian_2001} 
and half-Cauchy \shortcite{gelman_prior_2006}, as well as half-normal. Parameter estimates were 
not sensitive to the choice of prior. For computational reasons (nicer tail behavior) half-normal priors 
are used for the final model presented below.}
 


\begin{figure}[htb]
\centering

\vspace{2cm}

\begin{tikzpicture}
\node[const] (name) {$\mathbf{G}_1$} ;
\node[obs, right=of name,  color=BlueGreen,  fill=white, text=black] (r1) {$\nu_{1}$};
\node[obs, right=of r1, color=BlueGreen,  fill=white, text=black] (r2) {$\nu_{2}$};
\node[obs, right=of r2, color=BlueGreen,  fill=white, text=black] (r3) {$\nu_{3}$};
\node[obs, right=of r3, color=BlueGreen,  fill=white, text=black] (r4) {$\nu_{4}$};
\node[obs, right=of r4, color=BlueGreen,  fill=white, text=black] (r5) {$\nu_{5}$};
\node[obs, right=of r5, color=BlueGreen,  fill=white, text=black] (r6) {$\nu_{6}$};
\node[const, right=of r6] (dots1) {$\hspace{.33cm}$};
\node[const, right=of dots1] (dots2) {$\hspace{.33cm}$};
\node[const, right=of dots2] (dots3) {$$};
\edge [-, color=DarkSlateGray, bend left = 5] {r1} {r2} ;
\edge [-, color=MidnightBlue, bend right = 5] {r2} {r3} ;
\edge [-, color=MidnightBlue, bend left = 5] {r3} {r4} ;
\edge [-, color=MidnightBlue, bend right = 5] {r4} {r5} ;
\edge [-, color=MidnightBlue, bend left = 5] {r5} {r6} ;
\edge [dashed, -, color=MidnightBlue, bend right = 5] {r6} {dots1} ;
\end{tikzpicture}
%
 \hspace{4cm} 
 %
\begin{tikzpicture}
\node[const] (name) {$\mathbf{G}_2$} ;
\node[obs, right=of name,  color=BlueGreen,  fill=white, text=black] (r1) {$\nu_{1}$};
\node[obs, right=of r1, color=BlueGreen,  fill=white, text=black] (r2) {$\nu_{2}$};
\node[obs, right=of r2, color=BlueGreen,  fill=white, text=black] (r3) {$\nu_{3}$};
\node[obs, right=of r3, color=BlueGreen,  fill=white, text=black] (r4) {$\nu_{4}$};
\node[obs, right=of r4, color=BlueGreen,  fill=white, text=black] (r5) {$\nu_{5}$};
\node[obs, right=of r5, color=BlueGreen,  fill=white, text=black] (r6) {$\nu_{6}$};
\node[const, right=of r6] (dots1) {$\hspace{.33cm}$};
\node[const, right=of dots1] (dots2) {$\hspace{.33cm}$};
\node[const, right=of dots2] (dots3) {$$};
\edge [-, color=MidnightBlue, bend left = 5] {r1} {r2} ;
\edge [-, color=SlateBlue, bend left=30] {r1} {r3} ;
\edge [-, color=MidnightBlue, bend right = 5] {r2} {r3} ;
\edge [-, color=SlateBlue, bend right=30] {r2} {r4} ;
\edge [-, color=MidnightBlue, bend left = 5] {r3} {r4} ;
\edge [-, color=SlateBlue, bend left=30] {r3} {r5} ;
\edge [-, color=MidnightBlue, bend right = 5] {r4} {r5} ;
\edge [-, color=SlateBlue, bend right=30] {r4} {r6} ;
\edge [-, color=MidnightBlue, bend left = 5] {r5} {r6} ;
\edge [dashed, -, color=SlateBlue, bend left=30] {r5} {dots1} ;
\edge [dashed, -, color=MidnightBlue, bend right = 5] {r6} {dots1} ;
\edge [dashed, -, color=SlateBlue, bend right=30] {r6} {dots2} 
\end{tikzpicture}
%
 \hspace{4cm} 
 %
\begin{tikzpicture}
\node[const] (name) {$\mathbf{G}_3$} ;
\node[obs, right=of name,  color=BlueGreen,  fill=white, text=black] (r1) {$\nu_{1}$};
\node[obs, right=of r1, color=BlueGreen,  fill=white, text=black] (r2) {$\nu_{2}$};
\node[obs, right=of r2, color=BlueGreen,  fill=white, text=black] (r3) {$\nu_{3}$};
\node[obs, right=of r3, color=BlueGreen,  fill=white, text=black] (r4) {$\nu_{4}$};
\node[obs, right=of r4, color=BlueGreen,  fill=white, text=black] (r5) {$\nu_{5}$};
\node[obs, right=of r5, color=BlueGreen,  fill=white, text=black] (r6) {$\nu_{6}$};
\node[const, right=of r6] (dots1) {$\hspace{.33cm}$};
\node[const, right=of dots1] (dots2) {$\hspace{.33cm}$};
\node[const, right=of dots2] (dots3) {$$};
\edge [-, color=MidnightBlue, bend left = 5] {r1} {r2} ;
\edge [-, color=SlateBlue, bend left=30] {r1} {r3} ;
\edge [-, color=SkyBlue, bend left=45] {r1} {r4} ;
\edge [-, color=MidnightBlue, bend right = 5] {r2} {r3} ;
\edge [-, color=SlateBlue, bend right=30] {r2} {r4} ;
\edge [-, color=SkyBlue, bend right=45] {r2} {r5} ;
\edge [-, color=MidnightBlue, bend left = 5] {r3} {r4} ;
\edge [-, color=SlateBlue, bend left=30] {r3} {r5} ;
\edge [-, color=SkyBlue, bend left=45] {r3} {r6} ;
\edge [-, color=MidnightBlue, bend right = 5] {r4} {r5} ;
\edge [-, color=SlateBlue, bend right=30] {r4} {r6} ;
\edge [dashed, -, color=SkyBlue, bend right=45] {r4} {dots1} ;
\edge [-, color=MidnightBlue, bend left = 5] {r5} {r6} ;
\edge [dashed, -, color=SlateBlue, bend left=30] {r5} {dots1} ;
\edge [dashed, -, color=SkyBlue, bend left=45] {r5} {dots2} ;
\edge [dashed, -, color=MidnightBlue, bend right = 5] {r6} {dots1} ;
\edge [dashed, -, color=SlateBlue, bend right=30] {r6} {dots2} 
\edge [dashed, -, color=SkyBlue, bend right=45] {r6} {dots3} 
\end{tikzpicture}
\vspace{.25cm}

\caption{The undirected graphs corresponding to $\kappa = 1$, $\kappa = 2$ and $\kappa = 3$. 
Each node represents a Congress. The dashed lines indicate that only part of the graph is shown 
due to space constraints.}
\label{fig:rw_undirected_graphs}
\end{figure}


%As alluded to above, modeling the temporal dependence between Congresses is accomplished via the choice of penalty matrix $\mathbf{P}$. To model temporal dependence such that the set of Congresses $\partial_t = \{C_{t-2}, C_{t-1}, C_{t+1}, C_{t + 2} \}$ provides information about $C_t$, an undirected form of a second order random walk or autoregressive prior is used to penalize deviations from this hypothesized trend. This corresponds to setting $\mathbf{P} = \mathbf{D} - \mathbf{A}$, where $\mathbf{A}$ is a symmetric matrix with $a_{ij} = 1$ if $C_j \in \partial_i$ and 0 otherwise, and $\mathbf{D}$ is a diagonal matrix such that $\forall i = j, \: d_{ij} = \sum_j a_{ij}$. It can be useful to conceptualize the neighbor relations described by each set $\partial_t$ as an undirected graph $G$ with vertices $V=\{C_t: t=46,\dots,106\}$ and edges connecting the vertices corresponding to neighboring Congresses.\footnote{Note that here the term ``neighbor" is not reserved only for adjacent Congresses, but rather any Congress in $\partial_t$ is considered a neighbor of Congress $t$.} The penalty matrix $\mathbf{P}$ then has a zero for each missing edge and the resulting multivariate normal distribution is a GMRF with respect to $G$.
%
%The matrices $\mathbf{A}$ and $\mathbf{D}$ are commonly referred to as the adjacency and degree matrices because encoded in $\mathbf{A}$ are all neighbor relationships (in this case a temporal relationship between Congresses) and the diagonal elements of $\mathbf{D}$ are the number of neighbors (the degree) of each vertex.
%
%To illustrate why this form of $\mathbf{P}$ captures these particular assumptions, consider $N$ measurements of a variable $x$, with each measurement made at one of $T$ evenly spaced points in time. For simplicity assume $N=T$ and that unit of $x$ is a unit of time.  The sequence of equally spaced time measurements $(x^{[t]})_{t=1}^T$ corresponds to a grid of points on a line.  Fahrmeir & Lang (2001) suggest several possible choices for a prior on a smooth function $f(x)$, the simplest of which is a first order random walk ($RW_1$) prior.  Under the $RW_1$ prior, the first differences
%
%$$\Delta_t = f(x^{[t]}) - f(x^{[t-1]})$$
%
%are treated as independent and identically distributed standard normal random variables. 
%
%While this formulation of the $RW_1$ prior is {\it directed}, conditioning also on $f(x^{[t+1]})$ -- one step into the future -- forms an undirected $RW_1$, where the neighbors of time $t$ are both $t-1$ and $t+1$.  The associated graph $G$ therefore has vertices $V=\{\nu_t : t=1,\dots,T\}$, each of which has two neighbors, with the exception of $\nu_1$ and $\nu_T$, which have one neighbor. The penalty matrix $\mathbf{P}$ corresponding to the $RW_1$ prior with equally with equally spaced observations is the tridiagonal matrix
%
%$$ \mathbf{P} = \mathbf{D} - \mathbf{A} =  \begin{bmatrix}
%1  	& -1 	& 		& 	& \\
%-1  	& 2 	& -1 		& 	& \\
%  	& -1 	& \ddots 	& \ddots	& \\
%  	&  	& \ddots 	& 2 	& -1\\
%  	&  	& 		& -1 	& 1\\
%\end{bmatrix}
%.
%$$
%
%The $RW_2$ prior used in this thesis is simply an extension of the $RW_1$ to the case where, in addition to $x^{[t-1]}$  and $x^{[t+1]}$, the measurements $x^{[t-2]}$  and $x^{[t+2]}$ are also considered neighbors of $x^{[t]}$. The construction of $\mathbf{P}$ as the matrix difference $\mathbf{D} - \mathbf{A}$ also allows the estimation of an optional scalar parameter $\gamma \in [0,1]$, which, as a coefficient on $\mathbf{A}$, can be interpreted as representing the strength of dependence (Rue and Held, 2005).\footnote{The resulting precision matrix $\mathbf{Q}= (\mathbf{D}- \gamma \mathbf{A})/\tau^2$ is the defining feature of the conditional autoregressive (CAR) model.  When $\gamma$ is fixed at 1, and thus $\mathbf{P}= \mathbf{D} - \mathbf{A}$, it is known as the intrinsic conditional autoregressive (ICAR) model.} In the results section it will be demonstrated that the degree to which the proposed model fits the data depends (slightly but non-negligibly) on the inclusion of $\gamma$. 


