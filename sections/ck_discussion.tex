\section{Discussion}
\label{ck_discussion}

\subsubsection{Different interpretations}

Although Figure~\ref{fig:ck_bias} appears to be showing two versions of the same plot with
slightly different estimates, Cox and Katz's results (obtained by maximum likelihood estimation) 
and the results from the Bayesian model must be interpreted differently. The former are point estimates 
and confidence intervals, whereas the full Bayesian model provides an estimate of the marginal posterior 
{\it distribution} for each parameter. That is, the estimates obtained from the Bayesian model are the 
distributions themselves; presenting the results as point estimates and intervals is simply convenient for 
making comparisons to Cox and Katz's results. 

In the classical approach taken by Cox and Katz, a single value is obtained for each parameter and 
assessed in relation to infinitely many confidence intervals constructed from infinitely many hypothetical 
datasets. While in many situations it is possible to imagine repeating an experiment (e.g. coin flipping), 
such an interpretation makes less sense in the context of the historical events of interest to this analysis. 
For each confidence interval obtained from Cox and Katz's analysis, all that can be said is that it either 
contains the (theoretical) true parameter value or it does not. This is a consequence of taking the 
philosophical position that the parameter is fixed while the data and interval bounds are 
random variables. 

On the contrary, from the Bayesian perspective the parameter is treated as a random variable while 
the data and interval bounds are fixed. As a result, the estimates from the  Bayesian model have a 
more intuitive interpretation. Conditional on the model and data, there is a $p\%$ probability that the 
(theoretical) true value of the parameter lies in the $p\%$ interval. 
