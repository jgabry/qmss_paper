\chapter{Introduction}
\label{introduction}

\citeA{wawro_designing_2014} highlights a divide between historians and quantitative political scientists in regards to the historical study of political institutions. To account for concerns voiced by qualitative researchers, Wawro and Katznelson propose a novel quantitative approach, rejecting ``... the idea that one must choose between historical depth and quantitative rigor" (p. 527).

Wawro and Katznelson emphasize several areas that the commonly employed quantitative methods fail to adequately address: temporality, periodicity, context and specificity.\footnote{Note that here periodicity is not used in the mathematical sense but rather refers how relationships between variables are influenced differently by break points in history at particular locations and times.}  To tackle these issues, they advocate the implementation of Bayesian semi-parametric mixed models, which, compared to the typical models used in political science, can not only better account for parameter variation and unobserved heterogeneity but also limit the instability of estimation when the ratio of parameters to data is large. Furthermore, the Bayesian framework requires the specification of prior distributions that can be tailored to reflect important assumptions or results from previous findings. In particular, priors like Gaussian Markov random fields can be used to model interesting temporal and spatial relationships. 

Wawro and Katznelson apply their method in two reanalyses of studies from the subfield of American Political Development (APD). While the results are promising, there is still much work to be done to support their larger claims regarding the value of such models for historical social scientific research in APD and beyond. There is also the practical problem of computation, a topic Wawro and Katznelson largely omit from their discussion. Wawro and Katznelson use software called BayesX, which is appropriate and very efficient for the particular replications they choose, but offers limited flexibility in the choice of priors, likelihoods, and link functions, limiting its utility for a broader range of applications \shortcite{bayesx_software}. 

To make the recommendations of Wawro and Katznelson applicable in the broader context of social scientific inquiry will require software with greater flexibility. \citeA{goodrich_designing_2012} advocates using the probabilistic modeling language and C++ library Stan, which is used in this thesis to develop a framework for applying the approach advocated in \citeA{wawro_designing_2014} to a more diverse range of models than is currently possible with BayesX and other software packages.  Without such a framework it will remain unclear whether the recommendations of Wawro and Katznelson can be extended beyond the confines of their particular analyses.  

Accounting for variation over time (and potentially space) entails estimating a large number of parameters. In many cases this results in a high parameter to data ratio, which presents greater challenges for maximum likelihood estimation -- e.g. under-identification -- than for the Bayesian approach advocated by Wawro and Katznelson. Moreover, not only can the Bayesian models described by Wawro and Katznelson and in this thesis overcome this problem, but they encourage reframing these problems as opportunities. Bayesian semi-parametric mixed models with historically relevant smoothing priors can allow for greater parameter heterogeneity and let the data play a larger role in guiding inferences. 

