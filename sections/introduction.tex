\chapter{Introduction}
\label{introduction}

In some sense, it is a tautology to assert that quantitative historical analyses in the 
social sciences should begin from the premise that parameters of interest may exhibit 
considerable variation over time. The entire enterprise makes little sense if it is assumed 
a priori that parameters will be constant -- parameter homogeneity should be discovered, 
not imposed. That is, there is a difference between hypothesizing that temporal variation 
in parameters is minimal and building a model that guarantees it. 

Nevertheless, there are many examples in the literature of models that presuppose 
parameters are invariant to time as well as models that do not explicitly make this 
assumption but lack the flexibility to discover variation if present.  Even methods that 
do attempt to allow for important variation often sacrifice some degree of statistical 
coherence, requiring other dubious assumptions or questionable treatment of the data.\footnote{
Part~\ref{cox_katz} of this thesis discusses one such example.}

\citeA{wawro_designing_2014} argue that the standard quantitative approaches for historical 
analysis in the social sciences are insufficient for addressing certain key features common to 
the most successful qualitative historical analyses. Citing the concerns of qualitatively-inclined 
political historians, Katznelson and Wawro contend that traditional methods lack attention to 
the nuanced ideas of temporality, periodicity, context and specificity.\footnote{Note that here 
periodicity is not used in the mathematical sense but rather refers how relationships between 
variables are influenced differently by break points in history at particular locations and times.}  
Rejecting ``... the idea that one must choose between historical depth and quantitative rigor," 
they propose an approach to the historical study of political institutions -- and historical social 
scientific inquiry in general --  that centers on relaxing structural assumptions and embracing 
parameter heterogeneity when appropriate (p. 527). Specifically, Wawro and Katznelson 
demonstrate that semi-parametric mixed models with historically relevant smoothing priors 
can accommodate greater parameter variation and let the data play a larger role in guiding 
inferences without sacrificing the stability of estimates as the parameters-to-data ratio becomes 
large.  

Wawro and Katznelson present these ideas in the context of two reanalyses of studies from 
the subfield of American Political Development. In this thesis another empirical analysis is 
conducted demonstrating the merits of Katznelson and Wawro's recommendations. Similar in 
content and style to the applied statistics literature, considerable space is given to a formal 
description of the proposed statistical models, as well as a discussion of computational 
implementations and challenges thereof. 

%
%Many of the finer details of the statistical and computational methods (and challenges) involved in applying Katznelson and Wawro's recommendations are reviewed and then demonstrated in a reanalysis of . 
%\citeA{cox_gerrymandering_2007}
%ADD CLOSING SENTENCE



