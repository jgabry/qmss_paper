\chapter{Introduction}
\label{introduction}

In some sense, it is obvious that historical analyses in the social sciences should rarely assume a priori that parameters will be constant over time. Wowever it is not uncommon for this assumption to made explicitly by researchers (or implicitly made for them by the statistical models they choose). Moreover, there are many examples in the literature of methods that attempt to allow for important variation but sacrifice some degree of statistical integrity, requiring dubious assumptions or questionable treatment of the data.\footnote{An example of such an analysis is presented in Chapter~\ref{cox_katz} of this thesis.}  

\citeA{wawro_designing_2014} argue that the standard quantitative approaches for historical analysis in the social sciences fail to account for important features common to successful historical analyses. In particular, they contend that the standard statistical analyses lack attention to the nuanced ideas of temporality, periodicity, context and specificity, citing the concerns of qualitatively-inclined political historians.\footnote{Note that here periodicity is not used in the mathematical sense but rather refers how relationships between variables are influenced differently by break points in history at particular locations and times.}  Rejecting ``... the idea that one must choose between historical depth and quantitative rigor," Wawro and Katznelson propose a Bayesian approach to the historical study of political institutions -- and historical social scientific inquiry in general --  that focuses on the importance of parameter heterogeneity (p. 527). Semi-parametric mixed models with historically relevant smoothing priors can allow for greater parameter heterogeneity and let the data play a larger role in guiding inferences without sacrificing the stability of estimates as the parameters-to-data ratio becomes large.  

Wawro and Katznelson present their recommendations in the context of two reanalyses of studies from the subfield of American Political Development (APD). This thesis adds another empirical example demonstrating the utility of their methods in addition to discussing some of the finer details of the statistical methods and computational challenges involved. 