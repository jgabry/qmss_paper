\begin{savequote}[0.4\linewidth]
\singlespacing
\noindent Clearly, there is a tension here, \\ \noindent between parsimony and realism \dots
\qauthor{\citeA<-->[p.~302]{jackman_bayesian_2009}}
\end{savequote}

\chapter{Introduction}
\label{introduction}

This thesis begins from the premise that parameter heterogeneity is fundamental to quantitative 
historical analysis. Such a view is not unanimous, and the a priori assumption of 
parameter invariance is a common feature of the statistical models traditionally employed for 
historical social scientific inquiry. Moreover, methods that do not explicitly entail this assumption 
often lack the flexibility to discover interesting variation if present. Even attempts to allow for parameter 
heterogeneity often sacrifice some degree of statistical coherence, requiring other dubious 
assumptions or questionable treatment of the data. 
These concerns are the focus of \citeA{wawro_designing_2014}, in which the authors argue that 
the standard quantitative approaches are insufficient for addressing 
key features common to the most successful qualitative historical analyses. 

Citing the misgivings of qualitatively-inclined political historians, Katznelson and Wawro 
contend that traditional methods lack sensitivity to the nuanced ideas of temporality, periodicity, 
context and specificity. Rejecting 
``... the idea that one must choose between historical depth and quantitative rigor," 
they propose an approach to historical social scientific inquiry that centers on relaxing structural 
assumptions and embracing parameter heterogeneity when appropriate 
\shortcite[p.~527]{wawro_designing_2014}. Specifically, 
Wawro and Katznelson demonstrate that semi-parametric structured additive regression 
with historically relevant smoothing priors can accommodate greater parameter variation 
and let the data play a larger role in guiding inferences without sacrificing the stability of 
estimates as the parameters-to-data ratio becomes large.  

Wawro and Katznelson present these ideas in the context of the historical study of political 
institutions and two reanalyses of studies from the subfield of American Political 
Development. In this thesis, another empirical analysis is conducted demonstrating the merits 
of their recommendations. Similar in content and style to the applied statistics literature, 
considerable space is given to a formal description of the proposed statistical models and 
a discussion of estimation using the probabilistic programming 
language and Markov chain Monte Carlo sampler Stan, which is presented as an alternative to 
the computational strategy implemented by Katznelson and Wawro. It is argued that Stan can 
significantly contribute to the effort promoted by Wawro and Katznelson by enabling full Bayesian 
inference for a richer collection of statistical models than previously possible. 
