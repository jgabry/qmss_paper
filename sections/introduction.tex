\chapter{Introduction}
\label{introduction}

In some sense, it is a tautology to assert that historical analyses in the social sciences should rarely assume a priori that parameters will be constant or exhibit only trivial variation over time. Nevertheless it is not uncommon for this assumption to be made explicitly by researchers (or for it to be made implicitly by the models they choose). Moreover, there are many examples in the literature of methods that attempt to allow for important variation but sacrifice some degree of statistical coherence, requiring dubious assumptions or questionable treatment of the data.\footnote{Part~\ref{cox_katz} of this thesis discusses one such example.}  

\citeA{wawro_designing_2014} argue that the standard quantitative approaches for historical analysis in the social sciences are insufficient for addressing certain key features common to the most successful historical analyses. Citing the concerns of qualitatively-inclined political historians, Katznelson and Wawro contend that traditional methods lack attention to the nuanced ideas of temporality, periodicity, context and specificity.\footnote{Note that here periodicity is not used in the mathematical sense but rather refers how relationships between variables are influenced differently by break points in history at particular locations and times.}  Rejecting ``... the idea that one must choose between historical depth and quantitative rigor," they propose a Bayesian approach to the historical study of political institutions -- and historical social scientific inquiry in general --  that centers on relaxing structural assumptions and embracing  parameter heterogeneity when appropriate (p. 527). Specifically, Wawro and Katznelson demonstrate that semi-parametric mixed models with historically relevant smoothing priors can accommodate greater parameter variation and let the data play a larger role in guiding inferences without sacrificing the stability of estimates as the parameters-to-data ratio becomes large.  

Wawro and Katznelson present these ideas in the context of two reanalyses of studies from the subfield of American Political Development (APD). This thesis discusses many of the finer details of the statistical methods and computational challenges involved (Part~\ref{lit_review}) and adds another empirical example (Part~\ref{cox_katz}) demonstrating the advantages of Wawro and Katznelson's program. 


