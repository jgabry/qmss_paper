\chapter{Introduction}

Wawro and Katznelson (2013) highlight a divide between historians and quantitative political scientists in regards to the historical study of political institutions. To account for concerns voiced by qualitative researchers, Wawro and Katznelson (henceforth W\&K) propose a novel quantitative approach, ``rejecting [. . . ] the idea that one must choose between historical depth and quantitative rigor" (W\&K, 2013).

W\&K emphasize several areas that the commonly employed quantitative methods fail to adequately address: temporality, periodicity, context and specificity.\footnote{Note that here periodicity is not used in the mathematical sense but rather refers how relationships between variables are influenced differently by break points in history at particular locations and times.}  To tackle these issues, they advocate the implementation of Bayesian semi-parametric mixed models.\footnote{Briefly, Bayesian semi-parametric mixed models can be thought of as extending Bayesian generalized linear models (GLMs) to allow the additive predictor to incorporate metric or spatially correlated independent variables with unknown nonlinear effects (Fahrmeir and Lang, 2000). }  Their claim is that, compared to the typical models used in political science, these models can not only better account for parameter variation and unobserved heterogeneity but also limit the instability of estimation when the ratio of parameters to data is large (W\&K, 2013). Furthermore, the Bayesian framework requires the specification of prior distributions that can be tailored to reflect important assumptions or results from previous findings. In particular, priors like Gaussian Markov random fields can be used to model temporal and spatial relationships between parameters. 

W\&K apply their method in two replications of studies from the subfield of American Political Development (APD). While the results are promising, there is still much work to be done to support their larger claims regarding the value of such models for historical social scientific research also outside of APD. Working with Wawro -- I?m currently his research assistant -- I hope to use my thesis to help complete some of this work. 

There is also the practical problem of computation, a topic W\&K omit from their discussion. W\&K use software called BayesX, which is appropriate and very efficient for the particular replications they choose, but offers limited flexibility in the choice of priors and likelihoods. To make the recommendations of W\&K applicable in the broader context of social scientific inquiry will require software with greater flexibility. Goodrich et al. (2012) advocates the Bayesian software Stan (under development at Columbia), and I aim to use Stan to develop a framework for applying W\&K's approach to a broader range of models than is currently possible with BayesX.  Without such a framework it will remain unclear whether the recommendations of W\&K can be extended beyond the confines of their replications.  

In Cox and Katz as well as the other replications I intend to work on, accounting for parameter variation over time (and often space) entails estimating a large number of parameters. In many cases this results in a high parameter to data ratio, which presents greater challenges for maximum likelihood estimation -- e.g. under-identification -- than for the Bayesian approach of W\&K. But the point is not only that the Bayesian models described here can overcome this problem, but moreover that they allow us to reframe the problems as an opportunity. Bayesian semi-parametric mixed models with historically relevant smoothing priors can allow for greater parameter heterogeneity and let the data play a larger role in guiding our inferences.   