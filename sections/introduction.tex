\chapter{Introduction}
\label{introduction}


Quantitative historical analyses in the social sciences should begin from the premise that 
parameters of interest may exhibit considerable variation over time. The entire enterprise 
makes little sense if it is assumed a priori that parameters will be constant -- parameter 
homogeneity should be discovered, not imposed. That is, there is a difference between 
hypothesizing that temporal variation in parameters is minimal and building a model that 
guarantees it. 
%Statistical modeling applied to the study of history is a double-edged sword. 
%It provides powerful techniques for learning from data, but also 

Nevertheless, there are many examples in the literature of statistical designs 
that presuppose parameters are invariant to time as well as models that do not explicitly 
make this assumption but lack the flexibility to discover variation if present.  Even methods that 
do attempt to allow for important variation often sacrifice some degree of statistical 
coherence, requiring other dubious assumptions or questionable treatment of the data.\footnote{
Part~\ref{cox_katz} of this thesis discusses one such example.}

\citeA{wawro_designing_2014} argue that the standard quantitative approaches for historical 
analysis in the social sciences are insufficient for addressing certain key features common to 
the most successful qualitative historical analyses. Citing the concerns of qualitatively-inclined 
political historians, Katznelson and Wawro contend that traditional methods lack attention to 
the nuanced ideas of temporality, periodicity, context and specificity.\footnote{Note that here 
periodicity is not used in the mathematical sense but rather refers how relationships between 
variables are influenced differently by break points in history at particular locations and times.}  
Rejecting ``... the idea that one must choose between historical depth and quantitative rigor," 
they propose an approach to historical social scientific inquiry that centers on relaxing structural 
assumptions and embracing parameter heterogeneity when appropriate (p. 527). Specifically, 
Wawro and Katznelson demonstrate that semi-parametric mixed models with historically relevant 
smoothing priors can accommodate greater parameter variation and let the data play a larger role 
in guiding inferences without sacrificing the stability of estimates as the parameters-to-data ratio 
becomes large.  

Wawro and Katznelson present these ideas in the context of the historical study of political 
institutions, and conduct reanalyses of two studies from the subfield of American Political 
Development. In this thesis, another empirical analysis is conducted demonstrating the merits 
of their recommendations. Similar in content and style to the applied statistics literature, 
considerable space is given to a formal description of the proposed statistical models and 
a discussion of estimation using the probabilistic programming 
language and Markov chain Monte Carlo sampler Stan, which is presented as an alternative to 
the computational strategy implemented by Katznelson and Wawro. It is argued that Stan can 
significantly contribute to the effort promoted by Wawro and Katznelson by enabling full Bayesian 
inference for a richer collection of statistical models than previously possible. 


%using Stan (which is not what K & W used) can significantly contribute to the effort they are promoting.  

%
%Many of the finer details of the statistical and computational methods (and challenges) involved in applying Katznelson and Wawro's recommendations are reviewed and then demonstrated in a reanalysis of . 
%\citeA{cox_gerrymandering_2007}
%ADD CLOSING SENTENCE



