\chapter[Empirical example: Bias and Responsiveness in Congressional Roll-Call Votes]{Empirical example \\[-20pt]\large Bias and responsiveness in congressional roll-call votes}
\label{cox_katz}

\vspace{-1cm}
\section{Background}
\label{ck_background}

To demonstrate the benefits of the kinds of methods proposed by \citeA{wawro_designing_2014}, 
we conduct a reanalysis of \citeA{cox_gerrymandering_2007}, a study of bias and responsiveness 
in congressional roll-call votes in the 46th through 106th US Congresses (1879--2000). 

In the legislative context of Cox and Katz's analysis, bias refers to an advantage for one party 
in the efficiency with which its votes translate into legislative victories. For example, consider 
a majority party with only a small advantage in the number of seats its members occupy. If the 
majority is well-organized and unified it may win a much larger proportion of votes than its slim 
seat advantage would typically suggest. Responsiveness is defined as the degree to which the 
majority party's rate of roll-call victories is sensitive to variations in its mean vote share 
\shortcite{cox_gerrymandering_2007}. 

Cox and Katz hypothesize and find evidence of persistent bias towards the majority party during 
the periods from 1889-1910 and 1961-2000, known as the period of ``czar rule" and the post-packing 
era, respectively. Applied to the same data used by Cox and Katz, the Bayesian hierarchical 
model provides some support for their hypothesis but suggests that weaker conclusions 
be made about bias towards the majority over the time periods in question.\footnote{On the whole, 
\citeA{cox_gerrymandering_2007} is an excellent paper and covers 
much more than the single analysis discussed in this thesis. The use of Cox and Katz's analysis here
is not intended to single it out for criticism. It simply happens to be a good candidate for applying the 
approach advocated by Wawro and Katznelson.} 

The rest of Part~\ref{cox_katz} is organized as follows. First, the data is described and visualized 
in \ref{ckdata}. Section~\ref{ck_stats} begins with a description of Cox and Katz's methods of analysis 
and proposes an alternative analysis using the Bayesian STAR models reviewed in Part~\ref{lit_review}. 
The second half of \ref{ck_stats} is dedicated to introducing the Stan language and C++ library for 
probabilistic programming used to estimate the model. Finally, in \ref{results_convergence_checking}  
Cox and Katz's results are compared to the results from the Bayesian analysis.\footnote{In addition, the fit of the STAR model to the data is assessed using graphical posterior predictive checks in Appendix E.}  
